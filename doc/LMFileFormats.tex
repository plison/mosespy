{\IRSTLM} supports three output formats of LMs. These formats have the
purpose of permitting  the use of LMs by  external programs.
\COMMENT{External
programs could in principle estimate the LM from an $n$-gram table before
using it, but this would take much more time and memory! So the best thing
to do is to first  estimate the LM, and then compile it into a binary format   
that is more compact and that can be quickly loaded and queried by the 
external program.
}

\subsection{ARPA Format}
This format was  introduced in DARPA ASR evaluations  to exchange LMs.
ARPA format  is  supported by most third party LM toolkit, like SRILM and KenLM.
It is  a text format which is rather costly in terms of memory. There is no limit to
the size $n$ of $n$-grams.

\subsection{qARPA Format}
This extends the ARPA format by including codebooks that quantize 
probabilities and back-off weights of each $n$-gram level. This format
is created through the command {\tt quantize-lm}.

The header {\tt qARPA} identifies this type of LM format.

\subsection{iARPA Format}
This is an {\em intermediate} ARPA format in the sense that each entry of the file
does not contain in the first position the full $n$-gram probability, but just its
smoothed frequency, i.e.:\\
\noindent
{\tt ...\\
f(z|x y) x y z bow(x y)\\
...
}

The header {\tt iARPA} identifies this type of LM format.

\noindent
Nevertheless, iARPA format is properly managed by the {\tt compile-lm} command
in order to generate a binary version or a standard ARPA version.


\subsection{Binary Formats}
Both ARPA and qARPA formats can be converted into a binary format 
that allows for space savings on disk and a much quicker upload of
the LM file.  Binary versions can be created with the command 
{\tt compile-lm}, that produces files with  headers {\tt blmt} or {\tt Qblmt}.

\noindent
Moreover, for an even faster access you can store the ngrams in an inverted
order (see Section~\ref{sec:inverted-lm});
the files with inverted-ordered ngrams have headers  {\tt blmtI} or {\tt QblmtI}.

